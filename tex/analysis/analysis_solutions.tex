\documentclass[answers,12pt]{exam}
\usepackage[utf8]{inputenc, xcolor}
\usepackage{hyperref}
\usepackage{amsmath,empheq}
\usepackage{float}
\usepackage{todonotes}

\newcommand{\increment}{\text{++}}

\title{Analysis Solutions}
\author{Ashwin Sreevatsa}
\date{April 2023}

\begin{document}

\maketitle

\section*{Quick Remarks}
This will be a solution page for problems in Terence Tao's Analysis 1 (Fourth edition).
The problems I'm adding here are those that I worked on and solved myself.
I don't expect to add all the problems in the textbook, but rather those that I find the most interesting.
I will try and do a few problems from each section though, just to make sure I'm actually understanding the concepts properly.

\section{Introduction}

\section{Starting at the Beginning: The Natural Numbers}

\textbf{Proposition 2.1.6:} 4 is not equal to 0 
\todo{Add column proofs.}

\begin{solution}
    \[
        \begin{aligned} 
            &4 = 3\increment \text{ (definition 2.1.3)} \\
            &3\increment \text{ is a natural number (axiom 2.1, axiom 2.2).} \\
            &3\increment \neq 0 \text{ (axiom 2.3)}
        \end{aligned}
    \]
\end{solution}

\textbf{Proposition 2.1.8:} 6 is not equal to 2
\begin{solution}
    \[
        \begin{aligned}
            &6=2\\
            \implies &5 \increment = 1 \increment \text{ (axiom 2.2, def 2.1.3)}\\
            \implies &5 = 1 \text{ (axiom 2.4)}\\
            \implies &4 \increment = 0 \increment \text{ (axiom 2.2, def 2.1.3)}\\
            \implies &4 = 0 \text{ (axiom 2.4)}\\
        \end{aligned}
    \]
    We know that this is false from prop 2.1, so $6 \neq 2$.
\end{solution}


\section{Resources}
Terence Tao: \href{https://terrytao.wordpress.com/books/analysis-i/}{Analysis 1}


\end{document}
