\documentclass[answers,12pt]{exam}
\usepackage[utf8]{inputenc, xcolor}
\usepackage{hyperref}
\usepackage{amsmath,empheq,amsfonts,amssymb}
\usepackage{float}
\usepackage{enumitem}
\usepackage[]{mdframed}
\usepackage{todonotes}
\usepackage[normalem]{ulem}
\newcommand{\suchthat}{\text{ s.t. }}


\title{Analysis Notes}
\author{Ashwin Sreevatsa}
\date{April 2023}

\begin{document}

\maketitle

\section*{Quick Remarks}
This doc contains my notes as I'm working through Terence Tao's Analysis 1 (Fourth edition).
It turns out that measure theory tends to use a lot of ideas and concepts from analysis.
My fundamentals with analysis are slightly non-existent so I'm working through this text to build a much stronger foundation.
I'm also working through the exercises in a separate file.


\section{Introduction}
Real analysis is the study of real numbers, functions, sequences, and series.
This is in contrast with complex analysis, which is the study of complex numbers, functions, etc, harmonic analysis, which is the study of Fourier series and Fourier analysis, and functional analysis, which is studies functions and their properties in infinite dimensional spaces.

The reason real analysis is worth studying is because there are a lot of rules and properties covered in topics like calculus that aren't explained from first principles.
For example, concepts like limits, derivatives, integrals, convergence, etc often have a lot of hand-wavy explanations when covered for the first time.
This is probably the correct way to handle these intro courses, but as you start to see more examples of functions, sequences, etc, it becomes clear that rules, properties, and theorems that we have been taught in a non-rigorous manner often have very specific conditions and assumptions behind them.
It's often very easy to come up with examples that break our intuition if we haven't studying the topic from a rigorous point of view.

Tao goes into painstaking detail describing a number of scenarios where intuition is insufficient to resolve strange paradoxes.
For example, consider a sequence like the following sum: $1 - 1 + 1 - 1 + 1 \dots$.
What is this sum equal to?
On one hand, you can group up the sums in the following way:
$(1 - 1) + (1 - 1) + (1 - 1) \dots = 0$ (maybe?).
But we can also do something similar like the following:
$1 + (-1 + 1) + (-1 + 1) \dots = 1$ (maybe?).
But you can also take $S = 1 - 1 + 1 - 1 \dots$.
By rearranging the infinite sum, we can find that $S = 1 - S$ and so $S = \frac{1}{2}$.
Clearly something is going wrong.

Another example might look like the following where $x$ is a real number.
$\lim_{m \rightarrow \infty} x^m$.
Consider $n+1 = m$.
Then $\lim_{m \rightarrow \infty} x^m = \lim_{n+1 \rightarrow \infty} x^{n+1}$.
But $n+1 \rightarrow \infty \implies n \rightarrow \infty$, so maybe we can say that $\lim_{n+1 \rightarrow \infty} x^{n+1} = \lim_{n \rightarrow \infty} x^{n+1}$.
But then we find a situation where $\lim_{m \rightarrow \infty} x^m =  \lim_{n \rightarrow \infty} x^{n+1} = x \lim_{n \rightarrow \infty} x^{n}$.
If we let $L = \lim_{m \rightarrow \infty} x^m $, we have $L = xL$.
Cancelling $L$ gives us that any real number $x$ is 1.
Something must be going wrong with how we are applying the rules of arithmetic with the rules of limits.

For more fun examples like these, I definitely recommend taking a look at the text directly.

A common theme here is that once you start working with infinities, many previously defined properties of numbers, operations, sequences, etc that hold fine in the finite case break down. 
To understand when rules will hold fine and when they break down (and require additional patching to work), a more fundamental, rigorous approach is needed.

\section{Starting at the Beginning: The Natural Numbers}
The first few chapters of the text construct number systems one at a time.
The eventual goal is to start with the natural numbers, move on to integers, then rationals, then reals, then complex. 
(Or $\mathbf{N} \rightarrow \mathbf{Z} \rightarrow \mathbf{Q} \rightarrow \mathbf{R} \rightarrow \mathbf{C}$). 
This chapter starts with the natural numbers $\mathbf{N}$, or all the non-negative integers $0, 1, 2, \dots$. 
Following this, the chapter then defines addition and multiplication on the natural numbers, along with a number of properties about these operations that we take for granted.

This chapter introduces natural numbers with the notion of the `increment' operation $++$ and the number 0.
It turns out that there are a number of surprising axioms needed to avoid problems like wraparounds (for example, preventing $4++ = 0$), ceilings ($4++ = 4$), etc.

One takeaway: when we are working with natural numbers and we are trying to prove some property, the first thing one should consider is a proof by induction.

As a side note, there was an interesting discussion on whether numbers should be thought of as abstract concepts or as connected with a real external concept.
For example, complex numbers often feel very pointless since they don't seem to have any tangible connection to numbers we use on a daily basis.
It seems like every `discovery' of new types of numbers (negative numbers, irrational numbers, complex numbers) resulted in a similar sort of problem:
originally, a number system was tied to some real external concept (like the number marbles someone could hold in their pocket), and if a new number system was discovered (irrational number of marbles?), this resulted in a lot of unnecessary confusion.
It's therefore easier to think of numbers as abstract concepts that don't need a concrete model to understand:
if you use a number system to model something in the real world, then you can attach the abstract number system to a real concept, but otherwise it is easier to remain abstract.

\section{Set Theory}
This chapter covered the fundamentals of set theory, starting with the set theory axioms and discussing Russell's paradox before constructing objects like functions, tuples, and Cartesian products from sets.
The chapter ended with a discussion on cardinality of sets.

\section{Integers and Rationals}
The general theme of this chapter was defining new operations (subtraction, division) and new number systems (integers, rationals).
Note however that there is a difficulty: to define the number system, one needs to define the respective operation to reach those numbers. 
For example, there's no way to reach negative integers from natural numbers using only addition and multiplication: subtraction is needed. 
But to define the true version of the subtraction operation, we need to have the integers (the natural numbers are not closed under subtraction).
To break out of this chicken and egg problem, we need to define a formal version of this operation only on the natural numbers before then defining the integers and then finally defining the true version of this operation.
At this point, we generally show that the formal operation and the true operation are equivalent.

One way to think about this is that defining the formal version of these operations is setting up scaffolding: once the building has been constructed, the scaffolding can be discarded. This `scaffolding' is apparently needed to avoid circular arguments where more advanced results are used to prove more primitive statements.

A common theme seems to be initial constructions of number systems getting consumed by future, more useful constructions (naturals superceded by integers superceded by rationals, which should be superceded by reals).

Most of the work of this chapter went into the above along with then proving that the operations are consistent and maintain their expected properties.
We also introduced absolute values and exponentiation.
The final section demonstrated that even though any gap between rationals contains additional rationals, there exist numbers for which there are no rational numbers ($\sqrt{2}$), opening the door to defining the real numbers.

\section{The Real Numbers}
It's in this chapter that the pace starts picking up.
The goal of this chapter is to define real numbers and their construction.
The motivation for why we want to do this is that many natural examples come up in geometry, trigonometry, etc where numbers may not necessarily be rational (for example $\sqrt{2}$, $\pi$, etc).
To do this, we need to define the limit operation and the formal version of this operation.

The core machinery for defining the real numbers is the Cauchy sequence.
Intuitively, this is a sequence of numbers whose members `eventually' get arbitrarily close to one another.
We initially start with a more restricted form of a Cauchy sequence only using rational numbers: eventually, we define Cauchy sequences using real numbers and show that the latter is a superset of the former.

The formal definition is the following:
If a sequence $(a_n)_{n=1}^{\infty}$ is a Cauchy sequence, then for all rational (or real) $\epsilon > 0$, there exists an integer $N \geq 0$ such that for all integers $j,k > N$, we have that $|a_j - a_k| < \epsilon$.
Or more succinctly:
\[
    \forall \epsilon > 0, \exists N \in \mathbf{N} \suchthat \forall j,k >N \suchthat j,k \in \mathbf{N}, |a_j - a_k| < \epsilon
\]
This is not trivial to understand: it takes Tao 2 sections to really pin down all the necessary propositions and lemmas to properly define the Cauchy sequence.

Some examples of Cauchy sequences:

\begin{itemize}
    \item \textbf{Example 0:}
    $1, 1, 1, 1, \dots$
    \item \textbf{Example 1:}
    $0.9, 0.99, 0.999, \dots$
    \item \textbf{Example 2:}
    $0.1, -0.01, 0.001, -0.0001, \dots$
    \item \textbf{Example 3:}
    $4, 982, -54, 11005, -842, 0, 0, 0, \dots$
    (followed by arbitrarly many zeroes)
\end{itemize}
Example 3 is a Cauchy sequence because we only care about the behavior `eventually'.

Non examples of Cauchy sequences:
\begin{itemize}
    \item \textbf{Non-example 0:}
    1, -1,  1, -1, 1, -1 \dots
    \item \textbf{Non-example 1:}
    1, 2, 3, 4, 5, \dots
    \item \textbf{Non-example 2:}
    1, 0, 1, 0, 1, 1, 1, 0, 1, 1, \dots
    (where the sequence will have zeroes at steadily increasing intervals: for example $a_n = 0$ if $n$ is a power of 2, and otherwise $a_n=1$)
\end{itemize}
Example 2 is not Cauchy because no matter how long the sequence goes, we will eventually find a zero.

We eventually define real numbers as the Cauchy sequences of rational numbers.
As previously, much of sections 5.3 are spent showing that the properties of algebra previously defined are still consistent.
One interesting note: it takes much more work to define division rigorously since we now have to say something about what it means for a Cauchy sequence of rational numbers to `not be zero' (since of course division by zero is undefined).
The following section discusses properties about the ordering of real numbers.

The next section introduces the notion of the least upper bound property, which is an advantage that real numbers have over rationals.
For any subset of the real numbers, there is a least upper bound of those subsetes that is a real number. 
(Note that this is not necessarily true for rationals:
Consider some set of numbers $1, 1.4, 1.41, 1.414 \dots$ that approaches $\sqrt{2}$.
The least upper bound is $\sqrt{2}$ but this is not rational, so the least upper bound on the rationals for this set does not exist.)
With this, we can define the supremum and infimum of sets as the least upper bound and the greatest lower bound, respectively.

The final section then begins to define properties of real exponentiation:
specifically, what does it mean to take $x^y$, if $x$ is real and $y$ is rational.
Note that we have to wait until chapter 6 to define exponentiation when $y$ is real.

Some tips when trying to write proofs involving $\epsilon$ and $\delta$ terms:
If $|x|<\epsilon$ for all $\epsilon >0$, $|x| = 0$.
If $x \leq y + \epsilon$ for all $\epsilon > 0$ then $x \leq y$ (which is often useful if we want to show that $x=y$ and it is easy to show that $y \leq x$ but not the reverse).

\section{Limits of sequences}
The goal of this chapter is to introduce and expand on limits for sequences.
At a high level, this section introduces sequences of real numbers, describes suprema and infima on real sequences, introduces a number of useful definitions and formalizations for limits of sequences (the limit itself, limit inferior, limit superior, limiting points, subsequences), describes the limits of some commonly found sequences, and completes the formalization of real exponentiation.

This section completes the job of the previous section: to replace the formal version of the limit operation with the actual limit operation.
What this means is that previously, we defined the limits and Cauchy sequences of rational numbers, but we want to be able to define limits and Cauchy sequences of real numbers as well.
The latter supercedes the former, so once this is complete, we don't need to make any distinction between Cauchy sequences on reals vs on rationals.
Following this, the standard limit laws are proven for real numbers (theorem 6.1.19).

The most important theorem in this chapter might be theorem 6.4.18: the completeness of reals.
What this is essentially saying is that there are no gaps in the reals, as there are with rationals.
For rationals, we can describe a Cauchy sequence that converges to no rational value (e.g.\ a sequence converging to $\sqrt{2}$).
For reals, any Cauchy sequence will converge.
This is a crucial distinction between the two number systems.

Another major theorem that is included in section 6.6 is the Bolzano-Weierstrass theorem (theorem 6.6.8).
This theorem basically says that for a bounded sequence, there is a subsequence that converges.
This should make some intuitive sense: if a sequence is bounded, there is not enough `room' to spread out indefinitely, so at least some fraction of the total space must have arbitrarily many elements of the sequence.
That is the subsequence that converges (of which there must be at least 1, though there may be more).
This is a precursor to notions of compactness and the Heine-Borel theorem that I expect to appear later on.

A third theorem/lemma that is quite important is the squeeze theorem.
This will allow us to find the limit of many types of functions.

The final section 6.7 completes the definition of real exponentiation.

\section{Series}
The purpose of this section is to develop a coherent theory of the limit of sequences.

This chapter begins with the finite series and defines it inductively before proceeding to prove a number of intuitive properties.
(This includes summation properties: lemma 7.1.4, well-definedness: prop 7.1.8, summation laws: 7.1.11, summation of cartesian products: lemma 7.1.13, fubini's theorem for finite series: corollary 7.1.14).
These might seem tedious or redundant, but the reason they are worth studying in depth is because many of them break down for infinite summations!
For example: it may be obvious that if you rearrange a finite series, that the summation will be the same.
However, this is not necessarily true for infinite series, even if the series as a whole converges!
(Note how strange this sounds: there are infinite series that converge to a sum, but if you rearrange their terms, they might still converge but to another value.
This is where the notion of conditional versus absolute convergence come up.)

The following section dives into summations of countably infinite terms.
This is formalized by taking the limit of a finite series: $\lim_{N \to \infty} \sum_{n=1}^N a_n$.
The remainder of this section discusses conditions to determine if a series will converge or diverge (zero test, absolute convergence test, alternating series test, telescoping series) as well as generally useful lemmas/propositions (series laws).
It's in this section where we're starting to see the full power of what analysis has to offer, as many of these conditions may not be obvious without defining finite and infinite series formally.

Section 7.3 covers summations of non-negative numbers.
It turns out that these often have interesting properties.
Since a summation of non-negative numbers will be monotonically non-decreasing, if it is bounded, it will converge (prop 7.3.1).
The Cauchy criterion is particularly interesting: given only a fraction of the elements in a sequence, we can determine if the whole series is convergent or not as well as give rough bounds to what value it converges to.
Using these new properties, we are able to say more about series like geometric series, harmonic series about when they converge/diverge.
These are fundamental series that show up often in math.

The goal of this section is to develop a theory of when we can rearrange infinite series and still get the same value.
It turns out that there are 2 useful lemmas regarding this.
Lemma 7.4.1 says that a convergent series of non-negative real numbers will always rearrange to the same sum, and lemma 7.4.3 says that an absolutely convergent series of real numbers will rearrange to the same sum.
Note the counterexample $\frac{1}{3} - \frac{1}{4} + \frac{1}{5} - \frac{1}{6}\cdots$.
It turns out you can rearrange this to any possible value.

The last section introduces the root and ratio test, which is useful for determining more limits.
Specifically, we can now compute $\lim_{n \to \infty} n^{1/n}=1$.

\section{Infinite sets}
Zorn's lemma is particularly interesting.
The key idea is that often we want to construct some maximal object and it turns out that this might take countably infinite or uncountably many steps.
Of course if this just takes finitely many steps, there is no problem (you can use induction to prove whatever you like).
However, these sorts of induction proofs can't help if you need to take countable or uncountably many steps.
The idea is that if we're able to construct some non-empty partially ordered set (poset) $X$ with each non-empty totally ordered subset (chain) $Y \subset X$ having some upper bound, then there is at least one maximal element of $X$.
Notice that we didn't say anything about the cardinality of $Y$: for all we care, $Y$ can be finite, countably infinite, uncountable and as long as it has an upper bound, we are happy.
See \href{https://gowers.wordpress.com/2008/08/12/how-to-use-zorns-lemma/}{Tim Gower's blog} for a great explanation of this.

\section{Continuous functions on R}
The goal of this chapter is to flesh out a theory of continuity and continuous functions.
There are a number of key takeaways from this chapter, including a brief intro to the Heine-Borel theorem on $\mathbf{R}$, formal definition of limits of a function, definition of continuous functions (along with equivalent formulations and other properties of continuous functions), maximum principle of compact continuous functions, intermediate value theorem, monotonicity, uniform continuity.

The key ideas of section 9.1 are the introduction of adherent points, closures, limit points, and bounded sets.
Adherent points are the set of points that `stick' arbitrarily close to sets.
For example, in the set $(0,1) \in \mathbf{R}$, $0$ and $1$ are adherent points.
(For the point at 0, for any $\epsilon > 0$, there exists $x \in (0,1)$ such that $|0-x|< \epsilon$).
Closures 


\end{document}
