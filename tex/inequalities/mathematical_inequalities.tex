\documentclass[answers,12pt]{exam}
\usepackage[utf8]{inputenc, xcolor}
\usepackage{hyperref}
\usepackage{amsmath,empheq,amsfonts,amssymb, amsthm}
\usepackage{float}
\usepackage{enumitem}
\usepackage[]{mdframed}
\usepackage{todonotes}
\usepackage[normalem]{ulem}
\newcommand{\suchthat}{\text{ s.t. }}

\theoremstyle{definition}
\newtheorem{definition}{Definition}[section]
\newtheorem{example}{Example}[section]
\newtheorem{theorem}{Theorem}

\title{Mathematical Inequalities Notes}
\author{Ashwin Sreevatsa}
\date{February 2023}

\begin{document}

\maketitle

\section*{Quick Remarks}
This document is meant to contain notes on various interesting inequalities that I come across.


\section{Inequalities}

\subsection{Cauchy-Schwarz}
This comes up very often in a huge variety of subjects.
\begin{theorem}[Cauchy-Schwarz Inequality]
    Let $u,v \in \mathbb{R}^n$. Then $\langle u, v \rangle \leq \lVert u \rVert_2 \lVert v \rVert_2 $
\end{theorem}

\begin{proof}
    \[
        \begin{aligned}
            \langle u, v \rangle^2 &= {\Big(\sum_{i=1}^n u_i v_i \Big)}^2 \\
            &= \sum_{i=1}^n u_i^2 v_i^2 + \sum_{i \neq j} u_i v_i u_j v_j \\
            &\leq \sum_{1 \leq i,j \leq n} u_i^2 v_j^2 && \text{(AM-GM inequality)}\\
            &= \lVert u \rVert_2^2 \lVert v \rVert_2^2
        \end{aligned}
    \]
    So we have $\langle u, v \rangle \leq \lVert u \rVert_2 \lVert v \rVert_2 $
\end{proof}

\end{document}
