\documentclass[answers,12pt]{exam}
\usepackage[utf8]{inputenc, xcolor}
\usepackage{hyperref}
\usepackage{amsmath,empheq}
\usepackage{float}
\usepackage{enumitem}

\title{Measure Theory Notes}
\author{Ashwin Sreevatsa}
\date{January 2023}

\begin{document}

\maketitle
\setcounter{section}{-1}

\section*{Quick Remarks}
This text will be my set of notes as I work through a couple of measure theory textbooks.
I plan to distill and summarize everything I am learning in a much more concise document in the future.
The primary book I will be studying is Terence Tao's Introduction to Measure Theory.
Solutions can be found in "measure\_theory\_solutions.tex".

\section{Preface}

The book begins by describing the general number system we'll be working with in measure theory. 
Essentially, it seems like there is some value in having $ + \infty$ as an actual number in the number system.
(I expect that we will discuss why this is necessary in much more depth once we actually begin to discuss what measures are.)

\begin{itemize}
    \item \textcolor{red}{CONFUSION}: Why is there a difficulty in having infinity and negative numbers at the same time? (Remark 0.0.1, page xii).
    Tao claims that this results in two different theories of measures: one on $[0,+\infty]$ and the other on $(-\infty, \infty)$.
\end{itemize}

There's an interesting asymmetry when we define $\infty$ as its own number: we have upwards continuity but not downwards continuity.
What if we wanted the reverse? 
Let's say $\infty \cdot 0 = \infty$. 
Then note that if $n \to \infty$, $n \cdot 0 \not \to \infty \cdot 0$.

\begin{itemize}
    \item \textcolor{red}{EXPECT} Monotone convergence theorem and fundamental convergence theorem.
\end{itemize}

\section{Chapter 1: Measure Theory}

\subsection{Chapter 1.1: Prologue: The problem of measure}
Measure theory is focused on measures. 
In 1, 2, 3 dimensions, measure corresponds to length, area, and volume respectively.
The core question we're trying to answer is how do we define a general, consistent notion of measures to sets in $n$-dimensional Euclidean spaces whose properties match up with our geometric intuitions.

For many such sets, we already have a good understanding of what their measure should be.
For example, we intuitively know how long the closed interval $[0,3]$ is, the area of a circle with radius 4, the volume of a tetrahedron of side length $k$, etc.
However, the real use cases of measure theory comes out when describing the measures of much more complicated sets. (What is the measure of the rational numbers between 0 and 1 inclusive: $\mathbf{Q} \cap [0,1]$? 

Some immediate use cases for why this is useful include calculating integrals of functions (these integrals are fundamentally tied to the measure 
'under the curve' of some function).
As it turns out, the standard notion of a Riemann integral is limited in the types of functions whose integral it can calculate: by adapting stronger notions of measures that are applicable to more types of functions, we are able to develop more robust notions of integration.
(This is a bit vague, but we will get more into the details soon.)

Briefly, I'd like to discuss a few of the desirable properties that we might expect to appear when defining a measure.
\begin{itemize}
    \item Boolean closure: If $E,F$ are measurable (if we're actually able to assign $E,F$ with a measure), then $E \cup F, E \cap F, E/F, E \triangle F$ are all measurable.
    Basically if we have 2 sets that are measurable, we should be able to perform any standard set operation on them and the resulting set should still remain measurable.
    \item Additivity: for two disjoint sets $E,F$, we have $m(E \sqcup F) = m(E) + m(F)$.
    This essentially means that if we divide up any set into two non-overlapping pieces, the sum of the measures of the smaller pieces should be the same as the measure of the entire original set.
    \item Null set has measure 0: $m(\emptyset) = 0$.
    A set with no elements should have no measure.
    \item Monotonicity: if $E \subset F$, $m(E) \leq m(F)$. 
    If one set falls entirely within another larger set, the smaller set cannot have a greater measure than the larger set.
    \begin{itemize}
        \item Notice that monotonicity and additivity imply subadditivity, or that $m(E \cup F) \leq m(E) + m(F)$, where $E,F$ aren't necessarily disjoint.
    \end{itemize}
    \item Translation invariance: $m(E+x) = m(E)$, where $E+x$ simply translates set $E$ by some vector $x$.
    In other words, if we have some shape on a coordinate plane, we shouldn't be able to change its measure by simply moving it around on the plane.
    \item The measure should match up with our notions of length, area, volume, etc.
    For example, the measure of a line segment should be the length of the segment, the measure of a square should be the length multiplied by the width of the square, etc.
    \item As many sets as possible should be measurable.
\end{itemize}

Defining a consistent and general notion of measures is a lot harder than it seems: for example, one way we might try to calculate measure of some body in real life might be to divide up the body into pieces, calculate the measure of each of the smaller pieces, and sum the total.\\ \\
However, notice that this immediately runs into problems: 
\begin{enumerate}
    \item Problem 1: a body consists of infinitely many points of measure 0, which is indeterminate. 
    \item Problem 2: spaces with the same 'number' of points can have different measures: consider
    $X=[0,1], Y=[0,2]$.
    'Obviously' it seems that $Y$ should have twice the measure of $X$, since its twice its length.
    Yet, there is a bijective function mapping the two sets: $f: X \rightarrow Y, f(a)=2a$.
    From this perspective, it appears that the two sets have the same number of points, and might therefore have the same measure.
    \item Problem 3: This might only seem like an issue with uncountable or infinitely sized sets, but apparently you can do this with a finite number as well (see Banach-Tarsky paradox).
\end{enumerate}

To develop a general theory of measures that handles these issues gracefully, we're going to work step by step.
The first 'measure' we define is the elementary measure, which is a weak notion that can't describe most sets in $\mathbf{R}^n$ that we care about.
(I call it a 'measure' because it doesn't technically satisfy the axioms of a true measure, but it serves as a building block for stronger measures).

Second, we define the Jordan measure that describes many more sets than the elementary measure.
(Technically these aren't true measures either :/ )
Much of introductory calculus implicitly uses this notion of measure when defining Riemann integrals.
However this notion too is limited for reasons we will discuss.

Third, we cover Lebesgue measures which then describe nearly all the sets that we care about in $\mathbf{R}^n$.
Indeed, depending on the axioms of set theory that we assume (specifically if we don't take the axiom of choice), we can show that all sets in $\mathbf{R}^n$ are Lebesgue-measurable.

Finally, we generalize measure spaces outside of $\mathbf{R}^n$ with abstract measure spaces.

\begin{itemize}
    \item \textcolor{red}{CONFUSION} What is the axiom of choice? 
    Why does it result in pathological sets?
    \begin{itemize}
        \item \href{https://math.stackexchange.com/questions/132007/why-is-the-axiom-of-choice-separated-from-the-other-axioms}{Some additional explanations on Stackexchange.}
        \item According to a theorem of Godel, it's often safe to use axiom of choice since for certain types of statements because we can guarantee that if it was proven using axiom of choice we can prove it without axiom of choice.
        \item How exactly does Banach-Tarsky paradox arise?
    \end{itemize}
    \item \textcolor{blue}{TODO:} How far can we go with our intuitive understanding of measures before we need to invoke any actual measure theory stuff? 
    \item \textcolor{blue}{TODO:} Equality except for measure zero sets?
    \item \textcolor{red}{CONFUSION:} How does measure theory actually resolve the problems mentioned in this section? 
    How does Jordan measure fail, and how does Lebesgue measure fix it? 
    What is the relation between these types of problems and the problems of conditional probabilities and "variables talking to one another" like Evan mentioned in the Napkin? 
    What sorts of problems can't be solved without measure theory?
    Why is $2^{\Omega}$ unsuitable? (anki)
    Additional questions and comments here: \href{https://docs.google.com/document/d/1jym4JZ-bqRNGxGtEEVBKtedlWqwca3IJHacje2UngdQ/edit#}{Evan Chen Napkin notes}
\end{itemize}

\subsubsection{Chapter 1.1.1 Elementary Measure}
To describe a measure, we need to first note which sets in $\mathbf{R}^n$ (or in an abstract measure space) are measurable.
After this, we then have to describe how to assign a measure to these sets.
After this, we then prove additional properties that the measure satisfies to demonstrate that this is a potentially useful construction of a measure.
Some of these propreties might include Boolean closure, uniqueness of the measure, translation invariance, additivity, monotonicity, etc.

We start by describing the elementary measure.
We declare all 'boxes' to be measurable.
In one dimension, this is just any line segment $[a,b]$.
In two dimensions, this is any rectangle $[a_1,b_1] \times [a_2, b_2]$.
In three dimensions, this is any rectangular prism $[a_1,b_1] \times [a_2, b_2] \times [a_3,b_3]$.
So on and so forth.

We set the measure to be the intuitive one: for some $d$-dimensional box $[a_1,b_1] \times ... \times [a_d,b_d]$, the measure is $(b_1-a_1)(b_2-a_2)...(b_d-a_d)$.

Using this as a starting point, we declare all sets in $\mathbf{R}^d$ that are finite unions and intersections of boxes as elementary sets.

Some of the following exercises describe the properties of the elementary measure (boolean closure, consistent measure, uniqueness).
There are further exercises in the text as well.

However, note that there are many sets that are not measurable with the elementary measure.
For example, most 2-dimensional shapes that we care about are not elementary measurable!
(Circles, triangles, most functions on closed intervals, boxes that aren't aligned with the coordinate axes, etc).
Roughly, this is because you cannot approximate sets that don't look box-like using elementary sets.
(This is a non-rigorous explanation: its worth taking the time to prove each of these individually.)
That being said, we will use what we have learned about elementary measures to construct more useful measures.

Tao presents a potential notion of measure with 
\begin{equation}
m(E) := \lim_{N \rightarrow \infty}\frac{1}{N^d}\#(E\cap \frac{1}{N}Z^d)    
\end{equation}
This is a slightly strange formulation, but a diagram might clear this up a lot:

% TODO: Add the image back once it has been added to the github repo.
% \begin{figure}[H]
%     \centering
%     \includegraphics[scale=0.3]{diagrams/Lemma1.1.2NaiveMeasureIdea.jpg}
%     \caption{This diagram may not actually be the exact formulation described, but it should capture the essence.}
%     \label{fig:my_label}
% \end{figure}

The reason why this doesn't hold is that there is no reason to expect the limit of this should hold.
For example, consider any fractal: the measure under this definition will never converge. 
Vitali set is another example, probably.
Additionally it wouldn't satify translation invariance.\\
Some properties this does satisfy: $m(\bigcup E_i) \leq \sum_i m(E_i)$, $m(E+x) = m(E)$ for all elementary sets $E$ (sets that can be measured without needing to take the limit) and $x\i$
\subsubsection{1.1.2 Jordan Measure}
For defining new measures, there are 3 important pieces: what sets are measurable, how is a measure assigned to a set, and what desirable properties are satisfied by the measure.
As a quick preview:
\begin{itemize}
    \item The Jordan measure is assigned to a set by using an approximation of elementary sets.
    \item Jordan measurable sets consist of many more sets than the elementary sets: 
    in general, most 'smooth' and 'bounded' sets are Jordan-measurable.
    That being said, not all the sets we care about are Jordan measurable, which eventually leads to the formulation of the Lebesgue measure.
    \item Finally, it turns out that the Jordan measure satisfies a few key properties including boolean closure, finite additivity, translation invariance.

\end{itemize}
The core idea of the Jordan measure is that we can essentially approximate the measure of some body using the elementary sets.

We essentially define a Jordan outer measure and a Jordan inner measure of a set.
A Jordan outer measure of a set is the following: 
\[
    m^{\star, J}(E) = \inf_{B \supset E, \text{ B is elementary}} m(B)
\]
Loosely, what this means is that to find the Jordan outer measure of a set $E$, we find the smallest elementary set $B$ that contains the set $E$.
The measure of $B$ provides a sort of upper bound on the Jordan measure of a set.
(The reason we use infimum here instead of minimum is to note that while there may not be a minimum, in the limit $m(B)$ should tend towards some minimum).

We can do the same for the inner measure:
\[
    m_{\star, J}(E) = \sup_{A \subset E, \text{ A is elementary}} m(A)
\]
This Jordan inner measure is a lower bound on the Jordan measure of a set.

If the two quantities are equal ($m_{\star, J}(E) = m^{\star, J}(E)$), we say that a set is Jordan measurable.
The Jordan measure is then $m(E) = m_{\star, J}(E) = m^{\star, J}(E)$.

At this point, the textbook goes into depth to describe many of the properties of Jordan measurable sets.
I highly recommend anyone looking to get a much stronger intution about the properties of Jordan sets to work through the exercises described in the book.
Exercise 1.1.6 describes the intuitive properties of measures that Jordan measures satisfy (including Boolean closure, non-negativity, monotonicity, finite additivity, translation invariance, etc).

The textbook also describes the types of sets that are Jordan measurable.
Some examples of Jordan-measurable sets: continuous functions on closed domains, compact convex polytopes.
In general, 'smooth' and 'bounded' sets tend to be Jordan measurable.

Some non-examples of Jordan-measurable sets: unbounded sets, the set of rational numbers $\mathbf{Q} \cap [0,1]$, the set of irrational numbers $\mathbf{R/Q} \cap [0,1]$.
In general, sets that have many 'holes' or have a 'fractal-like' structure will not be Jordan-measurable.
(This is because the Jordan outer and inner measures will not be able to converge to the same value).

\begin{itemize}
    \item \textcolor{blue}{TODO} Do a few more problems maybe?
    \item \textcolor{blue}{TODO} Add stuff from 1.1 to anki (topological boundary, examples of measurable vs non-measurable sets, riemann integral, Darboux integral, p.1.1.22, p1.1.25, problem solving strategies)
    \item \textcolor{red}{CONFUSION} I still don't understand the connection between the Riemann integral and Jordan measure.
    \item \textcolor{blue}{TODO}: Do problem 1.1.15, 1.1.18, 1.1.19
    \item \textcolor{red}{CONFUSION}: why does 1.1.18 imply that a set and its closure are not Jordan measurable if they have the same outer measure?
\end{itemize}

\subsubsection{1.1.3 Connection with the Riemann integral.}
Quick sketch of what the Riemann integral looks like: TODO
\begin{itemize}
    \item Riemann integrals define integrals in the form of the areas of piecewise functions as the width goes to 0
    \item Darboux integrals defines integrals in the form of areas of piecewise functions that upper and lower bound a function.
    Supremum of lower bound, infimum of upper bound.
    \item Problem 1.1.23 The main reason that continuous functions on bounded domain are Riemann integrable has to do with the fact that continuous functions on compact domains are uniformly continuous.
    The rest should be fairly straightforward to prove.
    \item \textcolor{blue}{TODO}: 1.1.21, 1.1.25, 1.1.26  
\end{itemize}

\subsection{1.2 Lebesque measure}
As mentioned previously, the primary reason that we want to develop Lebesgue measures is because there are many sets that we care about that are not Jordan measurable.
In addition to the examples from the previous section, we can also show that even if we have some Jordan measurable sets $E_1,E_2, ...$, it turns out that the union $\bigcup_{n=1}^{\infty} E_i$ and intersection $\bigcap_{n=1}^{\infty} E_i$ are not necessarily Jordan measurable (see exercise 1.2.1, also related: exercise 1.2.2). 

There are several parts to developing a theory of Lebesgue measures.
As always with measures, we want to know what sets are measurable, how a measure is assigned to these sets, and what desirable properties are satisfied by this measure.

The way we're going to do this is by first describing a Lebesgue outer measure defined on every possible set in $\mathbf{R}^n$ (or in other words, where the domain is the power set of $\mathbf{R}^n$).
As it turns out, this outer measure will not satisfy some properties that we would hope to see in a measure.
(Namely, the Lebesgue outer measure doesn't satisfy the property of additivity: if $E,F$ are disjoint, we can't necessarily say that $m^{\star}(E \sqcup F) = m^{\star}(E) + m^{\star}(F)$. (\textcolor{blue}{TODO} make sure to show this)

But if we restrict the domain to the 'Lebesgue-measurable' sets, we're able to get many desirable properties that we are looking for.
This will be how we define the Lebesgue measure.
Note that there is a distinction between the Lebesgue measure and Lebesgue outer measure: the latter can be applied on more sets than the former but as a result some useful properties that the former possesses is missing from the latter.

\subsubsection{Properties of Lebesgue outer measures}
We first define how the Lebesgue outer measure is assigned to the set.

Note that we can write the Jordan measure as 
\[
    \begin{aligned}
        m^{\star,(J)}(E) &= \inf_{U \supset E, U \text{ elementary}} m^{\star}(U)\\
        &= \inf_{B_1\cup...\cup B_k \supset E: B_i \text{boxes}}|B_1|+...+|B_k|
    \end{aligned}  
\]
The reason why this works is because any elementary set can be written as the finite union of boxes.
(So $U = \bigcup_{i=1}^k B_i$)

We can define the Lebesgue outer measure as something similar but instead using countably infinite many sets $B$:
\[m^{\star}(E):=\inf_{\bigcup_{i=1}^{\infty} B_i \supset E: B_i \text{boxes}} \sum_{i=1}^{\infty}|B_i|\]
Just to be explicit, recall that elementary sets only supported finite unions, not countably infinite many unions.
So this modification is not trivial.

We now know how we apply the Lebesgue outer measure on sets.
We now declare that all subsets of $\mathbf{R}^n$ are measurable.
(As it turns out, doing this breaks some valuable properties we would like: the Lebesgue measure will repair this later on by restricting the sets that are measurable to only those for which these desirable properties are not broken.)

We are immediately able to find that the Lebesgue measure for any countable set is 0.
Consider $\epsilon/2^n$ trick or the degenerate boxes method. 
(\textcolor{blue}{TODO})

Exercise 1.2.3 describes some axioms that the Lebesgue outer measure has.
Lemma 1.2.6 shows that the Lebesgue outer measure of an elementary set agrees with the elementary measure of the set.

\begin{itemize}
    \item \textcolor{blue}{Anki} Pointwise convergence, uniform convergence, $\epsilon/2^n$ trick, tonelli's theorem
    \item There seems to be interesting differences between finite unions and countable unions that lead to qualitatively different types of objects.
    What about uncountably many infinite?
    \item Countable many sets doesn't add power to inner measure due to subadditivity?
    This leads to Caratheodory criterion for Lebesgue measurability?
    \item Lebesgue measures can be efficiently contained in open sets with small error measure? 
    (Similar to Jordan measurable sets being contained in elementary sets with small error measure?)
    \item Littlewood's principles
    \item Discussion on the motivation for Lebesgue measures, when they work and when they don't and why they generally work fine (non-Lebesgue measurable sets are pathological and generated from axiom of choice, without this axiom all sets are Lebesgue measurable).
    \item We also need to show that Lebesgue measures satisfy the desirable axioms 
    \item finite additivity can break down for non-measurable sets? What?
    \item examples of measurable and non-measurable Lebesgue sets (compact vs non-compact)
    \item Def 1.2.2 Lebesgue measure
    \item \textcolor{blue}{TODO} Remark 1.2.8: not sure why $\overline{U}$ contains $[0,1]$.
    \item Intuitively, why does the finite $\rightarrow$ countably infinite elementary sets from Jordan to Lebesgue give us so much more power?
    \item Exercise 1.2.3, lemma 1.2.6, lemma 1.2.9, lemma 1.2.11
\end{itemize}


\begin{itemize}
    \item \textcolor{blue}{TODO}: Why exactly can you do this? 
    Why finite? 
    Finite but infimum seems to behave like countable, does countable but infimum behave like uncountable?
    Why does countable add additional power over finite boxes?
    \item \textcolor{red}{CONFUSION} I'm slightly confused about the difference between the LEbesgue outer measure and the Lebesgue measure in general. 
    It seems like we only really use the former when proving stuff.
    \begin{itemize}
        \item Ok so it turns out there are some problems with uncountable subadditivity:
        we already have that Lebesgue outer measure of $\mathbf{R}^d$ is infinite yet $\mathbf{R}^d$ contains an uncountable number of measure 0 points.
        If we were to accept uncountable subadditivity, we would be forced to admit that $m(\mathbf{R}^d)=0$, which would contradict the Lebesgue measure. 
        The reason that countable subadditivity is fine is because it still remains consistent with finite additivity.
        \item Lebesgue outer measure has finite additivity (and countable additivity) only for measurable sets.
        The problem is that for non-measurable sets, we can get in a situation where 2 sets are sufficiently entangled.
        
    \end{itemize}
\end{itemize}

Interesting note: it's often harder to prove lower bounds on infimums than upper bounds, which makes sense.

There are some details involving Lebesgue outer measures and finite additivity:
it turns out that this property only really holds for measurable sets.
However, finite additivity seems to hold fine so long as there is some separation between the sets (as shown in lemma 1.2.5).

Lemma 1.2.6 demonstrates that the Lebesgue outer measure is consistent with the elementary measure.
This also means that the Lebesgue outer measure is bounded below by the Jordan inner measure and bounded above by the Jordan outer measure.

Lemma 1.2.9 describes the measures of 'almost-disjoint' boxes, or boxes whose interiors are disjoint.
This lemma turns out to be useful because open sets can be expressed as the countable union of almost disjoint boxes (lemma 1.2.11).
This immediately gets us a way to calculate the Lebesgue outer measure of open sets using the Jordan inner measure of that set.
Lemma 1.2.12 then gives a way to calculate the Lebesgue outer measure of any set $E$ by calculating the infimum of Lebesgue outer measures on open sets $U \supset E$.
(However interestingly, the reverse statement is false: the supremum of Lebesgue outer measures on open sets $U \subset E$ is not equal to the Lebesgue outer measure of the set

\subsubsection{Lebesgue measurability}
It seems like so long as we only consider Lebesgue measurable sets and countable operations, the Lebesgue measure satisfies all the intuitive properties that we are looking for.

Lemma 1.2.13 lists some examples of measurable sets.

\textcolor{red}{CONFUSION}: Axiom of choice can construct non-Lebesgue-measurable sets.
Why does this mean that we cannot generalize countable closure properties to uncountable closure properties?
Not sure I totally follow this.

Lemma 1.2.15 describes properties of Lebesgue measure.
\textcolor{red}{CONFUSION}: What is the difference between Lebesgue measure and Lebesgue outer measure?

\textcolor{red}{TODO}: Most of this section is still incomplete.
Rewrite this summary and answer a few of the questions. 

\subsubsection{Non-measurable sets}
The purpose of this section is to describe the notion of non-Lebesgue-measurable sets, give a few examples while showing that they aren't measureable by Lebesgue sets, and the providing some additional exercises.
Note that there is a theorem showing that without the axiom of choice, all sets in $\mathbf{R}^n$ are Lebesgue measureable, which implies that any proof showing that a set is non-measurable must use the axiom of choice.

Proposition 1.2.18: \textcolor{red}{TODO}: rewrite this section more clearly

This example is the first construction of a non-Lebesgue-measurable set.
The rough intuition basically says the following:
Consider the set $[0,1]$.
Take all of the rational numbers $C =\mathbf{Q} \cap [0,1]$.
What is the Lebesgue measure of this set?
The key contradiction happens if you notice that $[0,1] = \bigcup_{x} C + x: x \in \mathbf{R}$, for some uncountably many $x$.
Now, if $m(C) = 0$, then $m([0,1])=0$, but if $m(C)>0$, then $m([0,1]) = \infty$. 
There is no way to apply a consistent measure here. 

\subsection{The Lebesgue integral}
\textcolor{red}{TODO:} Key questions to answer:
\begin{itemize}[noitemsep]
    \item How does the Lebesgue integrals work?
    \item What can be done with Lebesgue integrals that can't be done with Riemann integrals?
    \item Are Lebesgue integrals consistent with Riemann integrals?
\end{itemize}


Rough notes:
\begin{itemize}[noitemsep]
    \item Absolute convergence: The summation condition of $\sum_{n=1}^{\infty}|c_n| < \infty$ implies that partial sums $\sum_{n=1}^{\infty} c_n$ converges.
    \item Conditional convergence doesn't have absolute values: $\sum_{n=1}^{\infty}c_n$ converging doesn't imply $\sum_{n=1}^{\infty}|c_n|$ also converges.
    \item Note: limit of unsigned infinite sum will always exist because the values are all non-negative so they cannot oscilate.
    As a result, they either approach some value or go to infinity.
    \item \textcolor{red}{TODO:} I'm not entirely sure why this analogy was introduced.
    \item \textcolor{red}{CONFUSION:} Follow up on why absolutely Lebesgue-integrable functions can be infinite on Lebesgue measure 0 sets?
    \item \textcolor{red}{CONFUSION:} There tends to be some strange asymmetry in certain formulations of concepts: 
    for example outer measures are more useful than inner measures and lower Darboux integrals might be more useful than upper Darboux integrals w.r.t. defining a Lebesgue measure.
    I'm pretty sure the origins of this is described in one of the earlier sections, but it's worth explicitly understanding why this is the case.
    \textcolor{red}{TODO}: actually play around with this and try and see why using the inner measure or the upper Darboux doesn't work.
    \item Currently assuming that we are going to a very similar thing to how we defined the Lebesgue measure: start with Darboux integral, add countably infinite pieces, get Lebesgue integral (similar to starting with Jordan measure, adding countably infinite sets and getting the Lebesgue measure).
    \item It seems like the current goal is to develop a theory of unsigned Lebesgue integrals followed by absolutely convergent Lebesgue integrals.
    In section 1.4, a bunch more convergence theorems get defined. 
    (Fatou's lemma, monotone convergence theorem, dominated convergence theorem, etc).
    \item Lebesgue integral, measures are an extension (a 'completion') on top of Riemann integrals/ Jordan sets.
    Interesting discussion about what this exactly means. (p50)
    \item \textcolor{red}{TODO:} I'm still confused about the notion of 'almost everywhere'. (Definition 1.3.5)
    I'd like to see examples of sets that are technically different but are equal almost everywhere.
    \item \textcolor{red}{CONFUSION} Why is the union of countably many null sets still a null set but the union of uncountably many null sets not necessariy a null set?
    I understand the examples of this but I'm curious about what the underlying reason for this might be.
    My current best guess is that uncountable unions of measurable sets isn't really defined (because it results in weird stuff like this).
\end{itemize}

\subsubsection{Integration of simple functions}
Simple functions on 1 dimension with 1 Lebesgue-measuarable set: $f=c_1 1_{E_1}$.
So basically $f(x) = c_1: x \in E_1$.
For example, let's consider $c_1 =3, E_1 = [0,1]$.
Then, \[f(x) = \begin{cases}
    3: 0 \leq x \leq 1 \\
    0: \text{otherwise}
\end{cases}\]

Simple functions on multiple dimensions with 1 Lebesgue measurable set will be a similar situation, except with a more complex body than $[0,1]$.
Simple functions on 1 dimension with multiple Lebesgue measurable sets should describe many more complicated graphs with more than 2 piecewise values.
Simple functions on multiple dimensions with multiple Lebesgue measureable sets should be a similar extension.
\textcolor{red}{TODO}: It would be worth taking the time to draw many more simple functions just to make sure you really know what's going on here.

When he says that the space $Simp(\mathbf{R}^d)$ is a complex vector space, he's saying that the functions themselves form a vector space: ie, you can have a field with the elements being functions.
Unsigned simple functions can take constants $c_k \in [0, \infty]$ meaning that the space of $Simp^{+}(\mathbf{R}^d)$ is a module (because $\infty$ has no multiplicative inverse and therefore the space cannot be a vector space).

Note on ``One easy consequence fo this is that if $f$ is a complex-vlaued simple function, then its absolute value is an unsigned simple function.''
If you decompose the measurable sets that make up a simple function such that the sets that make up the function are disjoint, the function will be constant on every set.
This means that you can just take the absolute value on every disjoint set and find an unsigned simple function.
\textcolor{red}{TODO}: I'm unsure what it means for a function to not be a simple function, and why that might imply that the absolute value is not an unsigned simple function.

\textcolor{blue}{Note}: it doesn't seem like the definition Lebesgue integral itself using countably infinitely many Lebesgue measurable sets: I think the countability comes from the sets itself.

Key insight: $\int_{\mathbf{R}^d}1_E(x)dx = m(E)$.
We can extend this to any simple functions very easily: $\int_{\mathbf{R}^d} g(x) dx = c_1 m(E_1) + ... + c_k m(E_k)$.
\textcolor{red}{TODO}: we still need to make sure that different representations of $g$ will still have the same measure (lemma 1.3.4).

Definition 1.3.5: 
\textcolor{red}{CONFUSION} A countable union of null sets will still be a null set, so if there is a countable set of properties that hold almost everywhere, they union of those properties will also hold almost everywhere.

Exercise 1.3.1 gives some properties of the simple unsigned integral.
Definition 1.3.6 introduces the notion of absolutely convergent simple integral.
Exercise 1.3.2 gives some properties of complex-valued simple integrals.

Functions that are defined almost everwhere (often due to divide by 0 errors) can often be extended to being defined everywhere by some methods.
This is apparently useful for analysis since many functions that people care about are only defined almost everywhere.
Functional analysis apparently does some interesting stuff here: considering functions that are almost equal as equivalence classes.
I didn't really understand remark 1.3.7.

\subsubsection{Measurable functions}
Piecewise constant integrals $\to$ Riemann integrals is analogous to unsigned simple functions $\to$ Lebesgue integrals.

Bird's eye view of this chapter:
\textcolor{red}{TODO}: Definition 1.3.8 gives a notion of unsigned Lebesgue integrals, and lemma 1.3.9 provides a bunch of equivalent formulations.
Exercise 1.3.3 then provides a bunch of examples of measureable functions.
More exercises are provided to give more examples of measureable functions.
Definition 1.3.11 then introduces the notion of complex measureability, and more exercises are provided to give more examples.

\subsubsection{Unsigned Lebesgue integrals}
\begin{itemize}
    \item Why are we learning this in the order we are?
\end{itemize}

Bird's eye view.
Exercise 1.3.10 introduces some properties of lower Lebesgue integrals.
The book then discusses why measurable funcitons are interesting: lower and upper Lebesgue integrals can match given some conditions (Exercise 1.3.11).
More exercises are provided to show additional properties, theorems, lemmas, etc.

\subsubsection{Absolute integrability}
Bird's eye view.
Def 1.3.17 covers the formal definition of absolute integrable and defines absolutely integrable Lebesgue integrals.
Remark 1.3.18 notes that there are a lot of ways you can attempt to extend the Lebesgue integral concept, but they tend to behave poorly with respect to other important operations.
Therefore it tends to be better to deal with these cases as they arise.
\textcolor{red}{TODO} It might be useful to actually play around with this and see what I get.
We discuss the $L^1(\mathbf{R}^d)$ space and the fact that we can define a distance function $d_{L^1}$ that makes this space a complete metric space: Riesz-Fischer theorem.
More properties of absolutely integrable Lebesgue measures are then defined.

\subsubsection{Littlewood's three principles}
Bird's eye view.
This section discusses three principles that describe a lot of the intuition behind Lebesgue measures and gives explanations and proofs of these intuitions.
Some theorems covered: thm 1.3.26 (Egorov's theorem), thm 1.3.28 (Lusin's theorem)

\section{Resources}
Terence Tao: \href{https://terrytao.files.wordpress.com/2012/12/gsm-126-tao5-measure-book.pdf}{Introduction to Measure Theory}, \href{https://math.solverer.com/library/terence_tao/an_introduction_to_measure_theory}{Solutions to exercises}\\
Sheldon Axler: \href{https://measure.axler.net/MIRA.pdf}{Measure, Integration, Real Analysis}\\
Donald Cohn: \href{https://www.fayoum.edu.eg/stfsys/stfFiles/273/1342/Measure%20Theory%20(2nd%20ed.)%20-%20Cohn,%20Donald%20L._5990.pdf}{Measure Theory}


\end{document}
