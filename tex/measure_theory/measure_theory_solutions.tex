\documentclass[answers,12pt]{exam}
\usepackage[utf8]{inputenc, xcolor}
\usepackage{hyperref}
\usepackage{amsmath,empheq}
\usepackage{float}


\title{Measure Theory Solutions}
\author{Ashwin Sreevatsa}
\date{January 2023}

\begin{document}

\maketitle
\setcounter{section}{-1}

\section*{Quick Remarks}
This will be a solution page for problems in Terence Tao's Introduction to Measure Theory.

\section{Preface}

\textbf{Exercise 0.0.1} If $(x_{\alpha})_{\alpha \in A} x_{\alpha}$
 is a collection of numbers $x_{\alpha} \in [0, +\infty]$
 such that $\sum_{\alpha \in A}x_{\alpha}< \infty$,
 show that $x_{\alpha}=0$ for all
 but at most countably many $\alpha$.
 \begin{solution}
     This is the same problem as why the sum of uncountably many positive numbers is infinite. (This isn't a real proof)
 \end{solution}

\section{Chapter 1: Measure Theory}

\subsection{Chapter 1.1: Prologue: The problem of measure}

\subsubsection{Chapter 1.1.1 Elementary Measure}

\textbf{Exercise 1.1.1} Boolean closure
\begin{solution}\\
Union is obvious.\\
Consider intersection of $E \cap F$ where $E = \bigcup_i E_i$, $F = \bigcup_j F_j$ and each $E_i, F_j$ are boxes.
Then clearly $E \cap F = \bigcup_{i,j}(E_i \cap F_j)$.
I claim that the intersection of any 2 boxes is also a box, which will show that $E \cap F$ is elementary. \\
We can divide any elementary set into a finite number of small enough boxes so that the set theoretic difference should still be an elementary set.\\
Same for symmetric difference.\\
Translation is obvious, easy to prove.
\end{solution}
\textbf{Lemma 1.1.2} Measure of an elementary set 
\begin{solution}
This seems fairly obvious and intuitive: proving (i) really only requires showing that the union of 2 boxes can be expressed as the finite union of disjoint boxes, and by induction the rest is done. 
This is a bit of a tedious proof but the basic idea is to split the boxes $A,B$ into $2^d$ mini-boxes, where $d$ is the number of dimensions, in such a way that $A\cap B$ is a minibox and each of $A,B$ have 7 disjoint mini-boxes. 
These 15 miniboxes will form a disjoint union of boxes that make up $E \cup F$. \\
(ii) Skipped because boring, tedious.
Tao's proof for this part is slightly strange
\end{solution}

\textbf{Exercise 1.1.3} Uniqueness of elementary measure
\begin{solution}
    From the problem formulation, we have the following properties of $m'$: 
    \[
        \begin{aligned}
        \forall E, m'(E) &\geq 0\\
        \forall E_1, E_2, m'(E_1\sqcup E_2) &= m'(E_1)+m'(E_2)\\
        \forall E, m'(E+x) &= m'(E): x \in R^d
        \end{aligned}
    \]
    Let's consider $c:=m'([0,1)^d)$. 
    Then $m'([0,\frac{1}{n})^d) = \frac{c}{n^d}$ by the second property and third properties: 
    we can split up $[0,1)^d$ into $n^d$ disjoint pieces whose total measure equals the measure of $[0,1)^d$.
    By translational invariance, all of those pieces must each have the same measure so we have $m'([0,\frac{1}{n})^d) = \frac{c}{n^d}$.\\
    Technically this proof is incomplete, as I still need to show that for any box $E$, $m'(E) = cm(E)$.
    This isn't trivial but it's mildly tedious and I didn't believe I'd be learning anything so I skipped it. 
    The interesting part here is dealing with boxes of real but not rational side lengths. 
    The intuition here is that we just need to ensure that they match the actual measure, such as by taking the limit of many rationally sized boxes.\\
    \href{https://math.solverer.com/library/terence_tao/an_introduction_to_measure_theory/exercise_1-1-3}{Additional solution here}
\end{solution}
\subsubsection{1.1.2 Jordan Measure}

\textbf{Exercise 1.1.6} Jordan measurability inherits properties of elementary measure
\begin{solution} TODO\\
\textbf{Boolean closure}
\begin{itemize}
    \item $E \cup F$: There must be some $A_E \subset E \subset B_E, A_F \subset F \subset B_F$ such that $m(B_E/A_E) < \frac{\epsilon}{2}, m(B_F/A_F)<\frac{\epsilon}{2}$. 
    Take $A_E \cup A_F, B_E \cup B_F$.
    We have the following:
    \[
        \begin{aligned}
            \epsilon &> m(B_E/A_E) + m(B_F/A_F)\\
            &\geq m(B_E/A_E \cup B_F/A_F) \\
            &\geq m((B_E \cup B_F)/(A_E \cup A_F))
        \end{aligned}
    \]
    where the second inequality holds from the property of finite subadditivity of elementary sets (measure of union of two sets is less than or equal the sum of measures of sets)
    and the last inequality holds from the monotonicity property (subsets have smaller measures than their supersets).
    \begin{itemize}
        \item \textcolor{red}{Alternate solution} \href{https://math.solverer.com/library/terence_tao/an_introduction_to_measure_theory/exercise_1-1-6}{here}. 
        They don't even bother working with the measures of set unions directly, they use a different but equivalent formulation of Jordan sets ($m(B)-m(A) < \epsilon$). 
        This results in much simpler proofs that don't bother using the finite subadditivity property at all.
    \end{itemize}
    \item $E \cap F$:
    \[
        \begin{aligned}
            m(B_E \cap B_F) - m(A_E \cap A_F) &= 
            (m(B_E)-m(B_E/(B_E/B_F))) - (m(A_E)-m(A_E/(A_E/A_F))) \\
            &= (m(B_E)-m(A_E)) - (m(B_E/(B_E/B_F)) - m(A_E/(A_E/A_F))) \\
            &\leq m(B_E)-m(A_E) \\
            &\leq \epsilon
        \end{aligned}
    \]
    \item $E/F$:
    \[
        \begin{aligned}
            m(B_E/B_F)-m(A_E/A_F) &=
            m(B_E)-m(B_E \cap B_F) - (m(A_E)-m(A_E \cap A_F)) \\
            &= (m(B_E)-m(A_E))-(m(B_E \cap B_F)-m(A_E \cap A_F)) \\
            &\leq m(B_E)-m(A_E)\\
            &\leq \epsilon
        \end{aligned}
    \]
\end{itemize}
\textbf{Non-negative:} $m(E) \geq m(A_E) \geq 0$.\\
\textbf{Finite additivity:} we have $A_E \subset E \subset B_E, A_F \subset F \subset B_F$. 
Then $m(A_E) \leq m(E) \leq m(B_E), m(A_F) \leq m(F) \leq m(B_F)$. 
Note that $A_E, A_F$ are necessarily disjoint and we must be able to find some $B_E, B_F$ that are also disjoint (if there is no such $B_E, B_F$, then $E,F$ cannot be disjoint).
Then we have that $m(A_E)+m(A_F) = m(A_E \sqcup A_F)$, $ m(B_E)+m(B_F) = m(B_E \sqcup B_F)$. But this means that $m(E)+m(F)=m(E \sqcup F)$, as desired. 
\end{solution}
\textbf{Exercise 1.1.7}
\begin{solution}\\
(1): Let's start with the simplest case here where $d=1$ and then work our way into higher dimensions. 
We want to show that $E = \{(x, f(x)): x \in B\}$ is Jordan-measurable.
So we want to show for all $\epsilon >0$, we want to find elementary sets $E_L,E_U$ such that $E_L \subset E \subset E_U$ such that $m(E_U/E_L)<\epsilon$. What will end up happening is that we will have $E_U$ cover $E$ while $E_L$ will be the null set.\\
Since $B$ is a compact set and $f$ is continuous, we know that $f$ is uniformly continuous: namely $\forall \epsilon_f >0, \exists \delta_f>0$ such that all points within a distance of $\delta_f$ from some $x$, the function will be within $\epsilon_f$ from $f(x)$.
In other words $||y-x||_2< \delta_f \implies |f(y)-f(x)| < \epsilon_f$.\\
Essentially what we want to do here is cover $E$ with a finite number of elementary sets (boxes) such that the total measure it takes up is less than $\epsilon$, for any possible $\epsilon$. This will be $E_U$\\
Let's take $k$ as the side length of the boxes $E_i$ that we want to use to cover $E$. 
If every point in $E_i$ is within $\delta_f$ from one another, then the functions at those points must be within $\epsilon_f$ from one another.
To achieve this, we can set $\delta_f > k$.
Note that until this point, we didn't use the fact that $d=1$ anywhere: we can generalize this expression to $\delta_f > \sqrt{dk^2} = k\sqrt{d}$ for $d$-dimensional boxes (which agrees with the previous expression for $d=1$).\\
The resulting measure should be upper bounded by $nk^d\epsilon_f$, where $n$ is the number of $k$-boxes needed to cover $B$. 
This is because we have $n$ boxes each with measure $k^d$ whose 'height' corresponding to the $d+1$ dimension (the dimension that function $f$ maps to) must be at most $\epsilon_f$.
In other words, $m(E_U) \leq nk^d\epsilon_f$.
$m(E_L) = 0$, where we just take the null set.\\
Now let's find $n$ and $k$.
Notice that if the dimensions of $B$ are $b_1, b_2, ..., b_d$, then $n = \lceil \frac{b_1}{k} \rceil \lceil \frac{b_2}{k} \rceil ... \lceil \frac{b_d}{k} \rceil$. 
For simplicity's sake, we can just set $b= \max(b_1,b_2,..., b_d)+1$ such that $n \leq \frac{b^d}{k^d}$.
Now we have that $m(E_U) \leq \frac{b^d}{k^d}k^d\epsilon_f = b^d \epsilon_f$. However, note that $b$ is a constant, and therefore we can set $\epsilon_f$ as small as we want. This means $m(E_u) < \epsilon$, and we're done.
To summarize the last few steps, we have the following:
\[
    \begin{aligned}
        m(E_U) &\leq nk^d\epsilon_f\\
        &\leq \ \lceil \frac{b_1}{k} \rceil \lceil \frac{b_2}{k} \rceil ... \lceil \frac{b_d}{k} \rceil k^d \epsilon_f \\
        &\leq \frac{b^d}{k^d}k^d \epsilon_f \\
        &= b^d \epsilon_f\\
        &< \epsilon
    \end{aligned}
\]

(2) I'll demonstrate that the set is Jordan measurable when $d=1$.
First notice that we still have uniform continuity.
Since the set is uniformly continuous, $\forall \epsilon > 0, \exists \delta > 0$ such that $\forall x,y \in \mathbf{R}, |x-y| < \delta \implies |f(x)-f(y)| < \epsilon$.
\end{solution}

\textbf{Exercise 1.1.8} Some examples where Jordan measurability is weak/breaks down.
\begin{solution}
    Quick sketch of proof of (1): all we want to do is show that the outer measures of $E$ and $\overline{E}$, where the latter is the closure of $E$, have the same outer measure.
    We only consider the closed outer measures $B, \overline{B}$ since closed and open elementary sets have the same measure anyways.
    The outer measure of the latter will provide an outer measure for the former.
    For the reverse direction, we want to show that any element $x$ on the closure of $E$ but not within $E$ itself will still be covered by $B$, the outer measure of $E$.
    The main interesting detail here seems to be the fact that for any such $x$, there is no $\epsilon >0$ such that the minimum distance $d(x, m)>\epsilon$.
    Since this is the case, the infimum of the distance from $B$ must be 0 and as a result $x \in B$.
    Therefore, the closure is within $B$.
    Since $B$ and $\overline{B}$ both cover $E, \overline{E}$, they must have the same Jordan outer measure.

    Quick notes on (4): For any elementary set that's a subset of $[0,1]^2$, there will exist an element in $\mathbf{Q}^2$ unless the set is empty. 
    So inner measure is 0.
    Any super set needs to contain basically all of $[0,1]^2$ meaning it has a measure 1.
    Therefore it is not Jordan measure-able.
    The same situation for the 'bullets'.
\end{solution}

\subsubsection{1.1.3 Connection with the Riemann integral.}

\textbf{Exercise 1.1.25} Area interpretation of Riemann integral
\begin{solution}
    Rough sketch:
    We can show this in the case that $f$ is non-negative.
    When $f$ can be negative, there must be some partition of the function into negative and non-negative domains.
    After this point, the same will hold. (probably).

    Assume that $f$ is non-negative.
    Let's assume that $E_+$ is Jordan-measurable.
    This means that $\forall \epsilon >0$ there is $A_+ \subset E_+ \subset B_+$ with $A_+, B_+$ being elementary sets and $m(B_+/A_+) < \epsilon$.
    We want to use the Darboux integral here because once we can show something is Darboux integrable, it is also immediately Jordan integrable.
    Essentially we can partition the domain where $A_+, B_+$ are finite piecewise functions on the domain.
    $A_+$ serves as the lower Darboux integral while $B_+$ serves as the upper Darboux integral.
    Since we know that $m(B_+/A_+) < \epsilon$, we can show that for all $\epsilon > 0$, we can find functions such that the upper and lower Darboux integrals differ by less than $\epsilon$.
    This is enough to finish the proof as this implies that the two Darboux integrals are equivalent and that this function is Darboux integrable.
    The measure is then equal to the Darboux integral. \\ \\

    Notes about strategies for how best to prove this:
    start as simple as possible (constant functions and boxes),
    progress to slightly harder sets (piecewise constant functions and elementary sets)
    progress to harder sets (Jordan measurable sets)
\end{solution}


\subsection{1.2 Lebesque measure}

\textbf{Exercise 1.2.1} Union and intersection of countable Jordan measurable sets is not necessarily Jordan measurable
\begin{solution}
    Sketch: Think of a counterexample where countable elements start behaving strangely with regards to measures.
    One should immediately think about $\mathbf{Q}$ and $[0,1]$.
    In this case, for unions, consider the union of rational numbers $q$.
    For intersections, consider the intersection of $q \cap [0,1]$.
    Both behave the same way.
\end{solution}

\textbf{Exercise 1.2.2 Pointwise convergence}
\begin{solution}
    Sketch: note that nearly all problems of these kinds should lead you to think about sets involving rational numbers.
    In this case, consider 
    \[
    f_n(x)=
        \begin{cases}
        1 \text{ if } x \in {q_i: 1 \leq i \leq n} \\
        0
        \end{cases}
    \]
    This sequences of functions will be Jordan measurable for all finite $n$ yet converges to a non-Riemann integrable function.
\end{solution}

\subsubsection{Properties of Lebesgue outer measures}

\textbf{Exercise 1.2.3 The outer measure axioms}
\begin{solution}
    \begin{enumerate}
        \item (Empty set) 
        We'd like to prove that $m^{\star}(\emptyset)$ = 0, where $m^{\star}$ is the Lebesgue outer measure.
        Recall that the definition of the Lebesgue outer measure of set $E$ is the following: \[m^{\star}(E):= \inf_{\bigcup_{n=1}^{\infty} B_n \supset E; B_1, B_2, \cdots \text{ boxes}} \sum_{n=1}^{\infty}|B_n|\]
        If $E=\emptyset$, then we can consider the set of boxes $B_i=\emptyset$.
        $|B_i| = 0$, so $m^{\star}(\emptyset) \leq \sum_{n=1}^{\infty}0 = 0$.
        Also recall that $m^{\star}$ is a non-negative function, which means that $m^{\star}(\emptyset) \geq 0$.
        This means that $m^{\star}(\emptyset) = 0$.
        \item (Monotonicity) Let's say that we have some set of boxes $B_1, B_2, \cdots : \bigcup_{n=1}^{\infty}B_n \supset F$.
        Then necessarily, $\bigcup_{n=1}^{\infty}B_n \supset E$.
        Therefore $m^{\star}(F) = \inf \sum_{n=1}^{\infty}|B_n| \geq m^{\star}(E)$.
        \item \textcolor{red}{CONFUSION} I actually don't understand this proof at all: I don't understand where/how axiom of choice is used.
        The proof in the solutions guide also doesn't explain how Tonelli's theorem for series is used in the proof properly.
    \end{enumerate}
\end{solution}
\textbf{Lemma 1.2.6} Lebesgue outer measure of elementary set agrees with elementary set.
\begin{solution}
    Postmortem:
    Didn't even consider Heine-Borel.
    I did consider the $\epsilon$ trick, but absent an actual strategy, I was mostly fumbling around.
    Also I was not aware that elementary sets could be open versus closed.
    Or that it really mattered.
\end{solution}

\textbf{Lemma 1.2.9} Outer measure of countable unions of almost disjoint boxes
\begin{solution}
What does interior, exterior actually mean?
I might need to try formalizing this a few different ways

We already have that it holds for a finite number of almost disjoint boxes.
What changes when we get from the finite case to countable case?
Any examples of this sort of proof?

Postmortem:
I was attempting to solve this problem using induction.
There seems to be some interesting discussions on why these sorts of proofs shouldn't generally work \href{https://math.stackexchange.com/questions/1872716/induction-countable-union-of-countable-sets}{here}.
The key idea seems to be that while induction can show that some property holds for any finite value, it doesn't necessarily mean that the proof holds for a countably infinite number of values, because there is no finite $n$ such that $n+1$ is countably infinite.
\end{solution}

\subsubsection{Lebesgue measurability}
\textbf{Lemma 1.2.13} Existence of Lebesgue measureable sets.
\begin{solution}
Attempt 1:

Before we begin, recall the definition of Lebesgue measureable sets:
If set $E$ is Lebesgue measureable, then for all $\epsilon>0$, there exists an open set $U \in \mathbf{R}^d$ with $U \supset E$ such that $m^{\star}(U/E) \leq \epsilon$. 
\begin{enumerate}
    \item \textbf{Every open set is Lebesgue measureable.}
    Clearly if $E$ is open, then just take $U=E$ which will guarantee that $U$ is open, $U \supset E$ and $m^{\star}(U/E)=m^{\star}(\emptyset) = 0 \leq \epsilon$.
    \item \textbf{Every closed set is Lebesgue measureable.}
    It seems like I don't have a good notion of what a closed set actually is.
    It seems like this would be different from part 5.
    
    Idea: closed sets in $\mathbf{R}^d$ are compact. 
    This means that we can find some open cover of nope this doesn't work.
    
    \item \textbf{Every set of Lebesgue outer measure zero is measureable.}
    One idea: if we can find an open set $E_2$ with measure 0 and we know that the union of $E$ (the original outer measure 0 set) and $E_2$ is also open, then $U=E \cup E_2$.
    \item \textbf{The empty set $\emptyset$ is Lebesgue measureable.}
    We can find open sets with arbirarily small Lebesgue measures that will always be a superset of the empty set.
    Essentially find some $U: m^{\star}(U) \leq \epsilon$.
    \item \textbf{If $E \subset \mathbf{R}^d$ is Lebesgue measurable, then so is its complement $\mathbf{R}^d/E$}
    \item \textbf{If $E_1, E_2, \cdots \subset \mathbf{R}^d$ are a sequence of Lebesgue measurable sets, then the union $\bigcup_{n=1}^{\infty}E_n$ is Lebesgue measurable.}
    \item \textbf{If $E_1, E_2, \cdots \subset \mathbf{R}^d$ are a sequence of Lebesgue measurable sets, then the intersection $\bigcap_{n=1}^{\infty}E_n$ is Lebesgue measurable.}
\end{enumerate}
\end{solution}

\subsubsection{Non-measurable sets}


\section{Resources}
Terence Tao: \href{https://terrytao.files.wordpress.com/2012/12/gsm-126-tao5-measure-book.pdf}{Introduction to Measure Theory}, \href{https://math.solverer.com/library/terence_tao/an_introduction_to_measure_theory}{Solutions to exercises}\\
Sheldon Axler: \href{https://measure.axler.net/MIRA.pdf}{Measure, Integration, Real Analysis}\\
Donald Cohn: \href{https://www.fayoum.edu.eg/stfsys/stfFiles/273/1342/Measure%20Theory%20(2nd%20ed.)%20-%20Cohn,%20Donald%20L._5990.pdf}{Measure Theory}


\end{document}
